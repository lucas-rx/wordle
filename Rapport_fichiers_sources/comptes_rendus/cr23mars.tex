\begin{center}
\begin{tabular}[c]{|l|l|}
    \hline
    Type de réunion & Réunion d'avancement \\
    \hline
    Lieu & Telecom Nancy \\
    \hline
    Horaires & 16h00 - 17h30 \\
    \hline
    Maître de séance & Lucas Rioux\\
    \hline
    Secrétaires & Lucas Rioux, Patrick O'Brien \\
    \hline
    Membres présents & Armand Londeix \\ & Patrick O'Brien \\ & Lucas Rioux \\ & Serge Téhé\\
    \hline
    Membres absents & Aucun \\
    \hline
\end{tabular}
\end{center}

\section*{Ordre du jour}

\textbf{
\begin{enumerate}
    \item Atouts et faiblesses des membres du groupe
    \item Début du site : premières directives
    \item Éléments de gestion de projet
\end{enumerate}
}

\tabto{1cm}Avant le début de la réunion, le groupe a créé un groupe Messenger, un serveur Discord pour pouvoir communiquer, un Google Drive pour partager les documents relatifs au projet et un document Overleaf pour rédiger le rapport.

\section*{1. Atouts et faiblesses des membres du groupe}
\tabto{1cm}Lucas prend la parole et questionne les membres du groupe sur leurs compétences en matière de programmation (Python, SQL, HTML, CSS, Flask, C). Tous les membres du groupe affirment maîtriser Python, SQL, HTML et Flask mais moins le CSS. Concernant le C, seul Serge est expérimenté. Serge a d’ailleurs plus d’expérience avec l’utilisation de git car un module y était entièrement consacré l’année scolaire précédente.\\ \\
\tabto{1cm}Lucas propose d’utiliser les branches de git pour collaborer ensemble sans gêne et ne pas devoir prendre le risque de perdre du travail en commitant directement sur la branche master. Les autres membres du groupe sont d’accord avec cette idée, Lucas propose que tout le monde se renseigne sur ce point d’ici la prochaine réunion (cf. to-do list).
\\

\section*{2. Début du site : premières directives}
\tabto{1cm}Il est décidé que l’application sera implémentée grâce à la bibliothèque Flask de Python. À première vue, seulement deux pages HTML seront nécessaires : une page de connexion et une page avec le jeu. Patrick évoque la question de la gestion des caractères spéciaux, Lucas précise que ce problème est facilement gérable en Python.\\ \\
\tabto{1cm}Vient ensuite la question de la récupération des mots du dictionnaire français : ils ont été extraits de la base de données Lexique (version 3.83) (http://www.lexique.org/) par Lucas, répertoriant l’intégralité des mots français. L’ensemble des mots a été extrait vers un fichier texte, puis les doublons ont été supprimés. Armand a, quant à lui, récupéré des fichiers texte sur le github de ScienceEtonnante avec des mots déjà sélectionnés et séparés dans différents fichiers selon leur longueur. Il sera décidé plus tard de quel dictionnaire utiliser.

\section*{3. Éléments de gestion de projet}
\tabto{1cm}Les éléments de gestion de projet à rédiger dès maintenant sont la matrice SWOT et le diagramme de Gantt. Lucas propose aux membres du groupe de télécharger le logiciel gratuit Gantt Project permettant une création et mise à jour facile du diagramme de Gantt. Patrick se désigne pour réaliser la matrice SWOT et le début du diagramme de Gantt pour la prochaine réunion.

\section*{TO-DO LIST}

\begin{center}
\begin{tabular}{|c|l|}
    \hline
    Armand Londeix & - Se renseigner sur l’utilisation des branches de git \\
    & - Mettre en place l’environnement (Flask), \\
    & implémenter le début de l’application \\
    \hline
    Patrick O'Brien & - Se renseigner sur l’utilisation des branches de git \\ 
    & - Rédiger la matrice SWOT \\
    & - Commencer le diagramme de Gantt \\
   \hline
    Lucas Rioux & - Se renseigner sur l’utilisation des branches de git \\
    & - Extraire les mots de la base de données Lexique, \\
    & en supprimer les doublons \\
    \hline
    Serge Téhé & - Se renseigner sur l’utilisation des branches de git\\
    & - Extraire les mots de la base de données Lexique, \\
    & en supprimer les doublons \\
    \hline
\end{tabular}
\end{center}

\tabto{0cm}\textbf{Prochaine réunion : Lundi 28 Mars à 13h00.}