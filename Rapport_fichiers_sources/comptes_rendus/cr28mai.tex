\begin{center}
\begin{tabular}[c]{|l|l|}
    \hline
    Type de réunion & Stand-up meeting \\
    \hline
    Lieu & Discord \\
    \hline
    Horaires & 14h00 - 15h00 \\
    \hline
    Maître de séance & Lucas Rioux\\
    \hline
    Secrétaire & Lucas Rioux \\
    \hline
    Membres présents & Armand Londeix \\ & Lucas Rioux \\ & Serge Téhé \\
    \hline
    Membres absents & Patrick O'Brien \\
    \hline
\end{tabular}
\end{center}

\section*{Ordre du jour}

\textbf{
\begin{enumerate}
    \item Avancement depuis la dernière réunion
    \item Ce qu'il reste à faire
\end{enumerate}
}

\section*{1. Avancement depuis la dernière réunion}

\begin{itemize}
    \item Armand a aidé Patrick pour le solveur, des erreurs l'empêchaient d'avancer. Il a aussi rédigé la partie \textbf{Solveur} du rapport, que Serge a relue.
    \item Serge a fait tourner son algorithme de recherche de meilleur mot d'ouverture pour 650 mots : il trouve le meilleur mot en 493 secondes, soit un peu plus de 8 minutes. Il a aussi terminé la fonction calculant l'entropie d'un mot.
    \item Lucas a terminé la fonction lisant dans le fichier texte \emph{wsolf.txt} la longueur du mot.
\end{itemize}

\section*{2. Ce qu'il reste à faire}

\tabto{1cm}Le plus important, à l'heure actuelle, est de trouver le mot d'ouverture pour chacune des longueurs du jeu. Il faudra aussi terminer le rapport et tester le code.

\section*{TO-DO LIST}

\begin{center}
\begin{tabular}{|c|l|}
    \hline
    Armand Londeix & - Optimiser le solveur : calculer le meilleur mot d'ouverture plus vite \\
    & - Document de conception : partie \textbf{Solveur} \\
    \hline
    Patrick O'Brien & - Choix du solveur \\
    \hline
    Lucas Rioux & - Mettre à jour le schéma BD \\
    & - Assembler les différentes fonctions écrites pour créer le \emph{main()} \\
    \hline
    Serge Téhé & - Optimiser le solveur : calculer le meilleur mot d'ouverture plus vite \\
    & - Compléter le rapport : partie \textbf{Solveur} \\
    \hline
\end{tabular}
\end{center}

\tabto{0cm}\textbf{Prochaine réunion : Dimanche 5 Juin à 14h00.}