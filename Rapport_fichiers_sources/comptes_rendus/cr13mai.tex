\begin{center}
\begin{tabular}[c]{|l|l|}
    \hline
    Type de réunion & Réunion d'avancement \\
    \hline
    Lieu & Discord \\
    \hline
    Horaires & 13h-13h30 \\
    \hline
    Maître de séance & Lucas Rioux\\
    \hline
    Secrétaire & Armand Londeix \\
    \hline
    Membres présents & Armand Londeix \\ & Lucas Rioux \\ & Serge Téhé\\ & Patrick O'Brien \\
    \hline
    Membres absents & Aucun \\
    \hline
\end{tabular}
\end{center}

\section*{Ordre du jour}

\textbf{
\begin{enumerate}
    \item Ce qui a été fait
    \item Todo list
\end{enumerate}
}

\section*{1. Ce qui a été fait}

\tabto{1cm}Armand a du mal en c à cause des listes et des structures pour stocker du texte mais est à l'aise avec l'idée du solveur, le rapport n'est pas encore fini. Quelques fonctions ont été crées et sont sur Gitlab cependant certaines sont inachevées. \\

Patrick a fait une partie du rapport mais avait des doutes sur le solveur, il a profité de la réunion pour poser des questions. \\

Lucas a avancé sur le solveur en python pour comprendre le principe du solveur en général. \\ \\
Serge a travaillé sur les structures pour stocker les mots.
 \\ \\

\section*{TO-DO LIST}
\tabto{1cm}Pour la prochaine fois nous allons faire ce qui n'a pas été fait pour la réunion et tous ensemble réfléchir sur le type de structure à implémenter pour les données avant de s'attaquer au solveur en tant que tel.
 \\ 
 Aussi nous allons par message nous tenir informé et essayer de déterminer avant ce week end la structure des données afin de commencer le solveur avant la prochaine réunion.
  \\ \\


\tabto{0cm}\textbf{Prochaine réunion : Mercredi 18 mai à 13h.}
