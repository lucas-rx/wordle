\begin{center}
\begin{tabular}[c]{|l|l|}
    \hline
    Type de réunion & Réunion d'avancement \\
    \hline
    Lieu & Discord \\
    \hline
    Horaires & 14h-15h \\
    \hline
    Maître de séance & Armand Londeix\\
    \hline
    Secrétaire & Armand Londeix \\
    \hline
    Membres présents & Armand Londeix \\ & Patrick O'Brien \\ & Lucas Rioux \\ & Serge Téhé\\
    \hline
    Membres absents & Aucun \\
    \hline
\end{tabular}
\end{center}

\section*{Ordre du jour}

\textbf{
\begin{enumerate}
    \item Ce qui a été fait depuis la dernière fois
    \item Ce qui reste à faire
\end{enumerate}
}


\section*{1. Ce qui a été fait}
\begin{itemize}
    \item Lucas a augmenté le nombre de listes par table afin d'augmenter la rapidité des algorithmes, il a d'autre part travaillé sur le solveur qui est maintenant fonctionnel.
    \item Serge a travaillé sur le rapport et notamment sur la partie solveur qui est aboutie. Il souhaitait travailler sur l'optimisation des algorithmes mais cette partie a déjà été faite par Lucas.
    \item Patrick n'a pas encore commencé la partie personnelle mais a commencé à y réfléchir, il souhaitait utiliser le solveur mais n'a pas réussi à l'implémenter.
    \item Enfin, Armand a travaillé essentiellement sur le rapport, sur la partie solveur ainsi que la partie personnelle et la partie sur le travail réalisé.
\end{itemize}


\section*{2. Ce qui reste à faire}
\tabto{1cm} Il faut finaliser le rapport, faire les bilans personnels ainsi que donner le nombre d'heures de travail et enfin faire le bilan du groupe. Aussi, la partie sur les complexités est à finir et des tests doivent être faits sur les algorithmes. Enfin, les meilleurs mots par longueur de mots doivent être trouvés. \\

\section*{TO-DO LIST}

À part Lucas et Serge qui peuvent finaliser le solveur, tout le monde doit faire le rapport et tester les algorithmes ainsi que trouver leurs complexités. \\ \\
\tabto{0cm}\textbf{Prochaine réunion : Mardi 7 juin à 15h00.}