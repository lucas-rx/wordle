\begin{center}
\begin{tabular}[c]{|l|l|}
    \hline
    Type de réunion & Réunion d'avancement \\
    \hline
    Lieu & Telecom Nancy \\
    \hline
    Horaires & 14h00 - 15h00 \\
    \hline
    Maître de séance & Lucas Rioux\\
    \hline
    Secrétaire & Lucas Rioux \\
    \hline
    Membres présents & Armand Londeix \\ & Patrick O'Brien \\ & Lucas Rioux \\ & Serge Téhé\\
    \hline
    Membres absents & Aucun \\
    \hline
\end{tabular}
\end{center}

\section*{Ordre du jour}

\textbf{
\begin{enumerate}
    \item Mise au point : comment fonctionne le solveur en théorie
\end{enumerate}
}


\section*{1. Mise au point : comment fonctionne le solveur en théorie}

\tabto{1cm}Le groupe se met d’accord sur comment fonctionne le solveur en théorie et règle les soucis de compréhension après avoir visionné la vidéo de 3Blue1Brown https://www.youtube.com/watch?v=v68zYyaEmEA. 

\tabto{1cm}Le joueur devra obligatoirement jouer le mot proposé par le solveur sans quoi il risque d’échouer (conformément au sujet), ce que Armand confirme.

\tabto{1cm}La question de la structure de données utilisée pour stocker le dictionnaire se pose : pour l’instant, seule une liste (chaînée ou contiguë) est envisagée, étant donnée que le groupe n’arrive pas à comprendre comment utiliser un arbre pour résoudre ce problème bien que cette option semble efficace du point de vue de la complexité.

\tabto{1cm}Patrick demande ce qui se passe si l’utilisateur rentre une mauvaise combinaison de couleurs : dans ce cas, le solveur échouera. De toute manière, le solveur ne connaissant pas le mot à trouver, il suffira simplement de le relancer.

\tabto{1cm}Patrick demande aussi s’il sera possible de jouer (sur l’interface Web donc) un autre mot que celui proposé par le solveur : pour l’instant, cette option n’est pas envisagée.


\section*{TO-DO LIST}

\begin{center}
\begin{tabular}{|c|l|}
    \hline
    Armand Londeix & - Coder en C pour le solveur \\
    & - Rapport : rédiger l’historique et les statistiques \\
    \hline
    Patrick O'Brien & - Rapport : rédiger l’introduction, l’expérience et les achievements \\ 
    \hline
    Lucas Rioux & - Mettre les CR dans le rapport \\
    & - Rapport : agencement du site \\
    \hline
    Serge Téhé & - Complexité : fonction d’évaluation de index.js \\
    & - Rapport : rédiger la partie sur la base de données \\

    \hline
\end{tabular}
\end{center}

\tabto{0cm}\textbf{Prochaine réunion : Vendredi 13 mai à 13h00.}