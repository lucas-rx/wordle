\begin{center}
\begin{tabular}[c]{|l|l|}
    \hline
    Type de réunion & Réunion d'avancement \\
    \hline
    Lieu & Telecom Nancy \\
    \hline
    Horaires & 12h30 - 13h20 \\
    \hline
    Maître de séance & Serge Téhé\\
    \hline
    Secrétaire & Serge Téhé \\
    \hline
    Membres présents & Patrick O'Brien \\ & Lucas Rioux \\ & Serge Téhé\\
    \hline
    Membres absents & Armand Londeix  \\
    \hline
\end{tabular}
\end{center}

\section*{Ordre du jour}

\textbf{
\begin{enumerate}
    \item Présentation des tâches accomplies
    \item Discussion à propos de la page "index.html"
\end{enumerate}
}

\section*{1. Présentation des tâches accomplies}
Au début de la réunion, Patrick pose des questions d'éclaircissement sur les tâches qu'il devrait accomplir qui concernent la fonction Achievement et l'insertion des descriptions dans la table.Lucas prend la parole et lui répond en lui faisant la remarque que l'idsucces en particulier et tous les identifiants devraient être supérieurs à 1. 
Patrick n'a donc pas terminé ses tâches, il devrait les finir pour la prochaine réunion.

Ensuite Serge prend la parole et présente la tâche qu'il devrait effectuer : le paramétrage de la grille Wordle; Il explique qu'il a réussit sa tâche et a aussi reglé les bugs au niveau du jeu car Lucas après avoir revérifié le jeu, s'est aperçu d'un bug. En outre, il affirme avoir entammé une autre tâche qui est la liaison des informations des joueurs et la base de donnée qu'il a aussi réussit et qu'il mettra sur le dépôt git après la réunion.

Enfin Lucas explique qu'il a aussi accompli toutes ses tâches à savoir la rédaction de certains comptes rendus en latex, le règlement des bugs au niveau de la page d'inscription.


\section*{2. Discussion à propos de la page "index.html"}

Serge affirme que le jeu ne fonctionne pas sur la route générant le fichier "index.html" à cause des formulaires qui ont été faits par Armand.

Lucas répond qu'il s'en occupera et va réamenager cette route de sorte à ce que le jeu fonctionne. 

Serge propose aussi de permettre qu'un utilisateur non connecté puisse pouvoir paramétrer la grille; Lucas répond: oui pourquoi pas? Serge réaffirme: oui ou non? Lucas se décide finalement et aquiesce cette idée.




\section*{TO-DO LIST}

\begin{center}
\begin{tabular}{|c|l|}
    \hline
    Armand Londeix & - Continuer la page Profile \\
    \hline
    Patrick O'Brien 
    & - Continuer les succès et la fonction Achievement \\
    \hline
    Lucas Rioux & - permmettre à l'utilisateur non connecté de paramétrer la grille \\
    & - Intégrer le jeu sur la page "index.html" \\
    \hline
    Serge Téhé & - Prendre en compte les points d'expérience dans la base de donnée \\
    & - Liaison de la base de donnée avec 
    les informations des joueurs  \\
    \hline
\end{tabular}
\end{center}

\tabto{0cm}\textbf{Prochaine réunion : Mardi 26 Avril à 19h00.}