\begin{center}
\begin{tabular}[c]{|l|l|}
    \hline
    Type de réunion & Stand-up meeting \\
    \hline
    Lieu & Discord \\
    \hline
    Horaires & 14h30 - 14h50 \\
    \hline
    Maître de séance & Lucas Rioux\\
    \hline
    Secrétaire & Armand Londeix \\
    \hline
    Membres présents & Armand Londeix \\ & Patrick O'Brien \\ & Lucas Rioux \\ & Serge Téhé\\
    \hline
    Membres absents & Aucun \\
    \hline
\end{tabular}
\end{center}

\section*{Ordre du jour}

\textbf{
\begin{enumerate}
    \item Parler du document de conception et de ce qui a été fait
\end{enumerate}
}

\section*{1. Parler du document de conception et de ce qui a été fait}

\begin{itemize}
    \item Tout le monde est d’accord avec ce qu’on a mis dans le document de conception et maintenant on attend la réponse de M. Festor ;
    \item Lucas a réalisé le Gantt ;
    \item Lucas pense qu’il faut faire une réunion tous les deux jours. Armand est d’accord mais préférerait des réunions assez courtes avec une réunion d’avancement longue moins fréquente. Tout le monde décide finalement de faire des réunions courtes (stand-up meetings) tous les deux jours de 10-15 minutes ;
    \item Tout le monde est d’accord que c’est trop tard d’ajouter des nouvelles fonctionnalités ;
    \item Lucas a commencé à apprendre le JavaScript. Armand pense que JavaScript est moins documenté en ligne du coup il faut minimiser le nombre de fichiers JavaScript. Il demande à Serge, qui a le plus d’expérience en JavaScript, s’il sait faire certains trucs en JavaScript ;
    \item Lucas demande comment récupérer le nombre de tentatives, Serge dit qu’il a une idée qu’il peut essayer.

\end{itemize}

\section*{TO-DO LIST}

\begin{center}
\begin{tabular}{|c|l|}
    \hline
    Armand Londeix & - Implémenter l'historique des parties \\
    & et des statistiques \\
    \hline
    Patrick O'Brien & - Créer les achievements \\
    & Rédiger les CRs en \LaTeX{} \\
    \hline
    Lucas Rioux & - Commencer le CSS \\
    \hline
    Serge Téhé & - Remplir le fichier .sql (pour créer la BD) \\
    & Modèle entité-association \\
    \hline
\end{tabular}
\end{center}

\tabto{0cm}\textbf{Prochaine réunion : Vendredi 22 Avril à 11h00.}