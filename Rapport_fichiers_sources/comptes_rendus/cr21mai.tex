\begin{center}
\begin{tabular}[c]{|l|l|}
    \hline
    Type de réunion & Réunion d'avancement \\
    \hline
    Lieu & Discord \\
    \hline
    Horaires & 17h00 - 18h00 \\
    \hline
    Maître de séance & Serge Téhé \\
    \hline
    Secrétaire & Serge Téhé \\
    \hline
    Membres présents & Patrick O'Brien \\ & Lucas Rioux \\ & Serge Téhé\\ & Armand Londeix\\
    \hline
\end{tabular}
\end{center}

\section*{Ordre du jour}

\textbf{
\begin{enumerate}
    \item Présentation des tâches accomplies et Discussion à propos du dictionnaire à choisir
    \item  Réglages bugs Armand
\end{enumerate}
}

\section*{1. Présentation des tâches accomplies}

Au début de la réunion, chaque membre a presenté les tâches qu'il a accomplit.

Lucas a avancé sur le rapport,  en particulier sur la partie préentation de l'interface web qu'il devrait finaliser. 

Ensuite Patrick prend la parole pour poser une question relative au fonctionnement du solveur, il demande si on utilise le dictionnaire des mots les plus fréquents ou le dictionnaire des mots autorisés. Tous les membres répondent qu'on travaille sur le dictionnaire des mots fréquents. En outre, Patrick propose une stratégie de jeu qui consiste à choisir le mot le plus fréquent parmi les possibilités lorsque le solveur ne trouve pas le mot au bout d'un grand nombre d'essais.

Finalement Serge prend la parole et présente l'implémentation de la structure de données qu'on utilisera dans notre solveur.

\section*{2. Réglages bugs Armand}
Pendant la réunion, Armand a présenté un bug qu'il rencontrait au niveau de la table de hachage et Serge l'a aidé à trouvé le bug.

\section*{TO-DO LIST}

\begin{center}
\begin{tabular}{|c|l|}
    \hline
    Armand Londeix & - Calcul de probabilités \\
    \hline
    Patrick O'Brien 
    & - Expliquer les règles du jeu dans le rapport \\
    \hline
    Lucas Rioux & - Fonction d'écriture et de lecture fichier wsolf.txt  \\
    \hline
    Serge Téhé & - Effectuer le calcul des entropies \\
    \hline
\end{tabular}
\end{center}

\tabto{0cm}\textbf{Prochaine réunion : Samedi 28 Mai à 14h00.}