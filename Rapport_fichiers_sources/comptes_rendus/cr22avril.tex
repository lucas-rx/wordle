\begin{center}
\begin{tabular}[c]{|l|l|}
    \hline
    Type de réunion & Stand-up meeting \\
    \hline
    Lieu & Discord \\
    \hline
    Horaires & 11h00 - 11h30 \\
    \hline
    Maître de séance & Lucas Rioux\\
    \hline
    Secrétaire & Patrick O'Brien \\
    \hline
    Membres présents & Armand Londeix \\ & Patrick O'Brien \\ & Lucas Rioux \\ & Serge Téhé\\
    \hline
    Membres absents & Aucun \\
    \hline
\end{tabular}
\end{center}

\section*{Ordre du jour}

\textbf{
\begin{enumerate}
    \item Parler de ce qui a été fait
\end{enumerate}
}

\section*{1. Parler de ce qui a été fait}

\begin{itemize}
    \item Lucas a bien développé le CSS du site ;
    \item Armand a travaillé sur l’historique et les statistiques. Il avait toujours une petite erreur avec la fonction ‘fetchone()’ mais on pense qu’il va la résoudre bientôt ;
    \item Patrick a fait une liste des achievements possibles et il a rédigé le premier compte-rendu en \LaTeX{}. Il a fait aussi une fonction démo pour calculer les achievements en utilisant les données dans la table ‘Partie’ mais on a décidé de mettre les achievements dans un autre table 'AchievementDesc' ;
    \item Serge a mis à jour le fichier databse.sql et a réalisé le modèle entité-association.
\end{itemize}

\section*{TO-DO LIST}

\begin{center}
\begin{tabular}{|c|l|}
    \hline
    Armand Londeix & - Implémenter l'historique des parties \\
    & et des statistiques \\
    \hline
    Patrick O'Brien & - Implémenter les succès \\
    & - Mettre leur description dans la BD \\

    \hline
    Lucas Rioux & - Rédiger les CRs en \LaTeX{} \\
    & - Finir le système d'inscription / de connexion \\
    \hline
    Serge Téhé & - Paramétrage de la grille selon \\
    & la longueur du mot et le nombre d'essais maximum \\
    \hline
\end{tabular}
\end{center}

\tabto{0cm}\textbf{Prochaine réunion : Dimanche 24 Avril à 14h00.}