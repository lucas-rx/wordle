\begin{center}
\begin{tabular}[c]{|l|l|}
    \hline
    Type de réunion & Réunion d'avancement \\
    \hline
    Lieu & Discord \\
    \hline
    Horaires & 13h00 - 13h40 \\
    \hline
    Maître de séance & Lucas Rioux\\
    \hline
    Secrétaire & Patrick O'Brien \\
    \hline
    Membres présents & Armand Londeix \\ & Lucas Rioux \\ & Serge Téhé\\ & Patrick O'Brien \\
    \hline
    Membres absents & Aucun \\
    \hline
\end{tabular}
\end{center}

\section*{Ordre du jour}

\textbf{
\begin{enumerate}
    \item Ce qui a été fait
\end{enumerate}
}

\section*{1. Ce qui a été fait}

\begin{itemize}
    \item Lucas et Patrick ont bossé sur la rapport;
    \item Armand a bossé sur des listes chaînées, il pensait que des listes chaînées ne peuvent pas avoir des mots, Serge n’est pas d’accord, il sait le faire. Lucas dit que on peut stocker les mots dans un arbre binaire mais c’est compliqué;
    \item Serge pensait qu’une table des hash codes des mots peut être utile. On met les mots avec le même hashcode dans la même liste. Les autres du groupe avaient un peu du mal à comprendre exactement ce qu’il voulait faire avec cette table mais on est d’accord que c’est la meilleure façon de stocker les mots. ;
    \item Serge a réussi à récupérer le dictionnaire en faisant ça. Il a fait aussi une fonction (en python) pour calculer l’entropie de chaque mot;
    \item Tout le monde a décidé de se renseigner sur le TP de hachage. 
\end{itemize}

\section*{TO-DO LIST}

\begin{center}
\begin{tabular}{|c|l|}
    \hline
    Armand Londeix & - Se renseigner sur les tables de hachage, rapport, fichier wsolf.txt (lire longueur des mots) \\
    \hline
    Patrick O'Brien & - Se renseigner sur les tables de hachage, rapport \\ 
   \hline
    Lucas Rioux & - Se renseigner sur les tables de hachage, rapport \\
    \hline
    Serge Téhé & - Se renseigner sur les tables de hachage \\
    \hline
\end{tabular}
\end{center}

\tabto{0cm}\textbf{Prochaine réunion : Samedi 21 Mai à 17h00.}