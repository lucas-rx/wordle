\begin{center}
\begin{tabular}[c]{|l|l|}
    \hline
    Type de réunion & Réunion d'avancement \\
    \hline
    Lieu & Telecom Nancy \\
    \hline
    Horaires & 13h00 - 14h00 \\
    \hline
    Maître de séance & Lucas Rioux\\
    \hline
    Secrétaire & Armand Londeix \\
    \hline
    Membres présents & Armand Londeix \\ & Patrick O'Brien \\ & Lucas Rioux \\ & Serge Téhé\\
    \hline
    Membres absents & Aucun \\
    \hline
\end{tabular}
\end{center}

\section*{Ordre du jour}

\textbf{
\begin{enumerate}
    \item Avancement depuis la dernière réunion et suppression de fichiers inutiles sur le git
    \item Élaboration de la base de données
\end{enumerate}
}


\section*{1. Avancement depuis la dernière réunion et suppression de fichiers inutiles sur le git}
\tabto{1cm}On compte prendre un nouveau dictionnaire, soit on garde celui d’Armand qui a l’avantage d’être trié par fréquence et constitue une grosse base de données, soit on prend celui de Lucas qui contient davantage de mots. De toute façon le changement du dictionnaire est relativement aisé et on pourra toujours revenir dessus. On décide donc de garder le dictionnaire d’Armand et de changer au cas où.

\section*{2. Élaboration de la base de données}
\tabto{1cm}On parle de la façon dont on organise le site, Lucas propose de créer un compte avant de pouvoir jouer, Armand pense qu’on peut jouer sans compte mais que dans ce cas là les achievements, les enregistrements de parties et autres fonctionnalités ne fonctionneraient pas, on va rester sur cette idée.

\tabto{1cm}Lucas pense qu’il faut supprimer la page succes qui ne sert pas à grand chose mais qui ne compromet pas le bon fonctionnement du site, il est décidé de revenir à ça plus tard.

\tabto{1cm}Pour tous se mettre d’accord, on fait sur un tableau le schéma du site ainsi que celui des bases de données.

\tabto{1cm}Pour la page de connexion Lucas propose de mettre juste un pseudo non pris, après réflexion on se demande s’il ne faut pas un mot de passe car certains joueurs pourraient jouer sur le compte d’autres.

\tabto{1cm}Pour la base de données, on créera 3 tables, une table utilisateur avec les noms d’utilisateur et les mots de passe et une table historique (utilisateur, nombre d'essai, mots utilisés,...), enfin une dernière table permettra de stocker les succès (victoires, victoires en moins de coups,...).

\section*{TO-DO LIST}

\begin{center}
\begin{tabular}{|c|l|}
    \hline
    Armand Londeix & - Continue la construction du site \\
    & - Page achievement, historique \\
    & - Base de données \\
    \hline
    Patrick O'Brien & - Document de conception \\ 
   \hline
    Lucas Rioux & - Mettre à jour la matrice RACI \\
    \hline
    Serge Téhé & - Gérer la fin du jeu \\
    & - evaluation de chaque lettre dans un mot \\
    & pour en associer des couleurs \\
    \hline
\end{tabular}
\end{center}

\tabto{0cm}\textbf{Prochaine réunion : Mercredi 6 Avril à 12h30.}