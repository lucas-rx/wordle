\subsection{Travail réalisé}

%Au cours de ce projet nous avons réalisé la conception et l'implémentation du jeu wordle en y intégrant  un système de compte utilisateur pour pouvoir enregistrer les parties, les scores,... Nous avons pu réaliser toutes les fonctions que nous avons imaginées au début du projet mais certaines idées qui nous venaient à la fin comme un mode de jeu contre la montre n'ont pu être implémenté faute de temps.\\
%Pour la partie solveur, nous avons créer un solveur en C qui permet effectivement lors de son utilisation de drastiquement réduire le nombre de coups utilisé. Le solveur est conforme à l'énoncé puisque qu'il trouve le mot optimal et qu'il renvoie la longueur du mot.

\tabto{1cm}Nous avons réussi à développer un jeu WORDLE paramétrable par le joueur (nombre d'essais, taille du mot, mode de jeu) avec un historique des parties et des fonctionnalités secondaires (succès, statistiques, expérience). Pour ce livrable, nous n'avons pas eu assez de temps pour créer un mode contre-la-montre, où le joueur aurait du deviner le maximum de mots en un temps donné.

\tabto{1cm}Le solveur est fonctionnel et réduit très vite l'ensemble des mots possibles pour converger vers la réponse : pour des mots comprenant de 4 à 8 lettres, la réponse est trouvée en 3 coups environ, et même en 2 coups pour des mots plus longs. Les livrables sont donc conformes aux attendus du projet.

\subsection{Bilan de l'équipe projet}

{\begin{center}
\begin{tabular}{|c|c|}
\hline Points positifs & - Implémentation des fonctions principales  \\
& - Renforcement des compétences de gestion de projet \\
\hline Points à améliorer & - Améliorer la communication au sein du groupe \\
& - Ne pas se concentrer sur les détails d'implémentation \\
& - Être plus efficace pendant les réunions \\
\hline Expérience collective & - Bonne impulsion de groupe \\
& - Utilisation de Gitlab \\
\hline
\end{tabular}
\end{center}}

\subsection{Bilan individuel}

\subsubsection{Armand Londeix} 
Dans la première partie du projet, je me suis impliqué dans l'interface Web, ainsi que dans la mise en place de certaines fonctions python, aussi cette partie m'a permis de consolider mes connaissances en HTML mais aussi de en m'impliquant dans la gestion du projet de m'améliorer en travail d'équipe, désormais je sais mieux comment renforcer l'organisation, la cohésion d'une équipe. \\
Dans la partie Solveur, le fait de coder en C m'a permis d'apprendre et de consolider mes connaissances dans un langage qui m'était nouveau, aussi je sais maintenant très bien utiliser les structures de données et coder en C n'est plus vraiment un problème. La collaboration avec mes camarades m'a permis de ne pas m'attarder sur certaines parties qui me posaient problèmes au début pour progresser dans mon travail. \\ 
Je garde un bon souvenir de ce projet tant sur le plan des connaissances que j'ai acquises que sur la gestion de projet et les relations avec mes camarades. \\ 
\subsubsection{Patrick O'Brien}
Ce projet m'a permis de développer ma programmation en algorithmes et en Web. J'ai appris à bien travailler en équipe. J'ai eu quelques problèmes dus à une mauvaise communication de ma part mais j'ai vu cela comme une opportunité de créer plus de cohésion dans le groupe. Il était très facile de travailler avec mes camarades et ils ont été très utiles pour résoudre les difficultés que j'ai rencontrées. En ce qui concerne le gestion de projet, je comprend bien l'importance d'une réunion efficace.
\\ 
 
\subsubsection{Lucas Rioux}

Ce projet m'a permis d'améliorer mes compétences en Python et en développement Web, notamment en CSS et en JavaScript ; et également de découvrir le langage C, qui m'a forcé à me montrer plus rigoureux. Certaines fonctions du projet m'ont mis à rude épreuve, comme la fonction éliminant les réponses impossibles dans le solveur.
J'ai également pu mettre en pratique mes connaissances en gestion de projet et je commence à comprendre comment travailler en équipe. Dans l'ensemble, je suis satisfait des autres membres de l'équipe ainsi que du travail effectué.

\subsubsection{Serge Téhé}
Ce projet m'a permi d'acquérir des connaissances plus poussées en algorithmique, en programmation Web et en langage C. J'ai pu développer tout au long du projet certaines aptitudes et réflexes en informatique et en gestion de projet qui me donnent ainsi une expérience de plus sur mon CV. Ce projet m'a aussi permis d'écouter les autres membres du groupe et de trouver un point commun de convergence des idées malgré nos divergences souvent très ostentatoires.

